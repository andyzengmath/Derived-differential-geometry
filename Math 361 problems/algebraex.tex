\documentclass[12pt]{amsart}
\setlength{\textheight}{8.5in} \setlength{\textwidth}{6.5in}
\oddsidemargin 0in \evensidemargin 0in


\usepackage{amsmath,amssymb,graphicx}
\usepackage[all]{xy}
\usepackage[bbgreekl]{mathbbol}
\usepackage{tikz-cd}
\vfuzz2pt % Don't report over-full v-boxes if over-edge is small
\hfuzz2pt % Don't report over-full h-boxes if over-edge is small
% THEOREMS -------------------------------------------------------

\newtheorem{thm}{Theorem}[section]
\newtheorem{cor}[thm]{Corollary}
\newtheorem{example}[thm]{Example}
\newtheorem{lem}[thm]{Lemma}
\newtheorem{prob}[thm]{Problem}
\theoremstyle{definition}
\newtheorem{defn}[thm]{Definition}
\theoremstyle{remark}
\newtheorem{rem}[thm]{Remark}
\numberwithin{equation}{section}
\newcommand{\bull}{\cdot}
\newcommand{\del}{\partial}
\newcommand{\Sq}{\mbox{Sq}}
\newcommand{\loc}{{\mbox{loc}}}
\newcommand{\supp}{\mbox{supp}}
\renewcommand{\del}{\partial}
\newcommand{\CL}{{\mathcal L}}
\newcommand{\cc}{{\mathbold c}}
\newcommand{\CC}{{\mathsf C}}
\newcommand{\CG}{{\mathcal G}}
\newcommand{\CP}{{\mathcal P}}
\newcommand{\CF}{{\mathcal F}}
\newcommand{\Cinfh}{S^\infty}
\renewcommand{\H}{{\cal H}}

\newcommand{\Z}{{\mathbb{Z}}}
\newcommand{\Q}{{\mathbb{Q}}}
\newcommand{\R}{{\mathbb{R}}}
\newcommand{\C}{{\mathbb{C}}}
\newcommand{\N}{{\mathbb{N}}}

\newcommand{\B}{{\mathcal{B}}}
\newcommand{\T}{{\mathcal{T}}}
\newcommand{\f}{\mathcal{F}}

\begin{document}
	\author{Qingyun Zeng}
\title{Group theory practice problems 1}
\curraddr{\textsc{Department of Mathematics,
	University of Pennsylvania, Philadelphia, PA 19104}}
\email{\tt{qze@math.upenn.edu}}


\maketitle
\tableofcontents	
\section{Basic definition}
\begin{prob}
Prove that if $G$ is an abelian group, then for all $a,b \in G$ and all integers
$n$, $(a\cdot b)^n = a^n\cdot b^n$.
\end{prob}
\begin{prob}
	If $G$ is a group such that $(a\cdot b)^2 = a^2\cdot b^2$ for all $a, b \in G$, show that
	$G$ must be abelian. 
\end{prob}
\begin{prob}
	If $G$ is a finite group, show that there exists a positive integer $N$ such
	that $a^N = e$ for all $a \in G$.
\end{prob}

\begin{prob}
	\begin{enumerate}
		\item If the group $G$ has three elements, show it must be abelian.
		\item Do part $(1)$ if $G$ has four elements.
		\item Do part $(2)$ if $G$ has four elements
	\end{enumerate}
\end{prob}

\begin{prob}
	 Show that if every element of the group $G$ is its own inverse, then $G$ is abelian.
\end{prob}

\begin{prob}
	If $G$ is a group of even order, prove it has an element $a \not= e$ satisfying
	$a^2 =e$. 
\end{prob}

\begin{prob}
	For any $n > 2$ construct a non-abelian group of order $2n$. (Hint:
	imitate the relations in $S_3$.)
\end{prob}

\section{Subgroups}
\begin{prob}
	If $G$ has no nontrivial subgroups, show that $G$ must be finite of
	prime order. 
\end{prob}

\begin{prob}
	\begin{enumerate}
		\item If $H$ is a subgroup of $G$, and $a \in G$. Let $aHa^{-1}=\{ aha^{-1}| h\in H\}$.
		Show that $aHa^{-1}$ is a subgroup of $G$.
		\item If $H$ is finite, what is the order of $aHa^{-1}$? 
	\end{enumerate}

\end{prob}

\begin{prob}
	Write out all the right cosets of $H$ in $G$ where
	\begin{enumerate}
\item $G = (a)$ is a cyclic group of order 10 and $H = (a^2)$ is the
subgroup of G generated by $a^2$. 
\item $G$ as in part $(1)$, $H = (a^5)$ is the subgroup of $G$ generated by $a^5$. 
	\end{enumerate}
\end{prob}

\begin{prob}
	If $a \in G$, define $N(a) = \{x \in G | xa = ax\}$. Show that $N(a)$ is a
	subgroup of $G$. $N(a)$ is usually called the {\bf normalizer} or {\bf centralizer} of
	$a$ in $G$.
\end{prob}

\begin{prob}
	 If $H$ is a subgroup of G, then by the {\bf centralizer} $C(H)$ of $H$ we mean
	the set $\{x \in  G | xh = hx \ \forall h \in H\}$. Prove that $C(H)$ is a subgroup
	of $G$. 
\end{prob}

\begin{prob}
	The {\bf center} $Z(G)$ of a group $G$ is defined by $Z(G) = \{z \in G | zx = xz\  \forall
		x \in G\}$. Prove that $Z(G)$ is a subgroup of $G$. Can you recognize $Z$ as
	$C( T)$ for some subgroup $T$ of $G$? 
\end{prob}

\begin{prob}
	If $H$ is a subgroup of $G$, let $N(H) = \{a \in G | aHa^{-1} = H\}$. Prove that
		\begin{enumerate}
			\item $N(H)$ is a subgroup of $G$.
			\item $H\subset N(H)$.
		\end{enumerate}
We call $N(H)$ the {\bf normalizer} of $H$ in $G$.
\end{prob}

\begin{prob}
	If $a \in G$ and $a^m = e_G$, prove that the order of $a$ divides $m$. 
\end{prob}

\section{Homomorphisms}
\begin{prob}
	Let $G$ be a finite abelian group of order $ord(G)$ and suppose the integer
	$n$ is relatively prime to $ord(G)$. Prove that every $g \in  G$ can be written
	as $g = x^n$ with $x \in G$. ({\sc Hint}: Consider the mapping $\phi :G \to G$
	defined by $\phi(y) = y^n$, and prove this mapping is an isomorphism
	of $G$ onto $G$. 
\end{prob}

\begin{prob}
	Let G be the dihedral group defined as $\{
	x,y |x^2 =e,\ y^n = e,\ xy =
	y^{-1}x\}$. Prove
	\begin{enumerate}
		\item The subgroup $N = \{e,y,y^2 ,\cdots,y^{n-1} \}$ is normal in $G$.
		\item That $G/N \simeq W$, where $W = \{1, -1\}$ is the group under
		the multiplication of the real numbers.
	\end{enumerate}

\end{prob}

\begin{prob}
	Prove that a group of order $9$ is abelian. 
\end{prob}

\begin{prob}
	If $G$ is a non-abelian group of order 6, prove that $G\simeq  S_3$.
\end{prob}

\begin{prob}
	 If $G$ is abelian and if $N$ is any subgroup of $G$, prove that $G/N$ is
	abelian
\end{prob}

\begin{prob}
	 Let $G$ be the group of all nonzero complex numbers under multiplication
	and let $\bar{G}$ be the group of all real $2 \times 2$ matrices of the form $
	\begin{pmatrix} 
		a & b \\
		-b & a 
	\end{pmatrix}$, where not both $a$ and $b$ are 0, under matrix multiplication. 
	Show that $G$ and $\bar{G}$ are isomorphic by exhibiting an isomorphism of
	$G$ onto $\bar{G}$. 
\end{prob}


\bibliographystyle{amsplain}
\begin{thebibliography}{10}

	\bibitem {A}  Artin, Michael. {\it Algebra}. 2nd ed. Boston: Pearson Prentice Hall, 2011.
	
	
	
	\bibitem {B} Armstrong, M. A. {\it Groups and Symmetry}. New York: Springer-Verlag, 1988.
	\bibitem {C} Aluffi, Paolo.{\it Algebra : Chapter 0}. Providence, R.I.: American Mathematical Society, 2009.
	
	
	
	
\end{thebibliography}
\end{document}